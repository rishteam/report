\chapter{摘要}

\section{背景簡介}

本專題是以 C++ 開發之 2D 遊戲引擎,開發者可以使用引擎提供之編輯器(RishEditor)編輯遊戲場景(Scene),並且可以對遊戲物件(Entity)附加遊戲邏輯(使用C++撰寫,並和編輯器一同編譯),
並支援 Batch Rendering \footnote{一種 Rendering 技巧,可支援同屏幕高達 100000 個 sprites} 、2D Lighting (Point Light, Ambient Light), Particle System, Constrain-based Physics (支援圓形、多邊形等)

\section{問題說明}

以往我們寫遊戲大部分都使用較為成熟的遊戲引擎,但往往都不能了解裡面的實作和內部的原理。透過本次專題,能夠讓開發者清楚了解到內部的實作,以及透過遊戲引擎來加速整個遊戲開發流程。

\section{實作結果}

成員們透過撰寫遊戲以及參考大型遊戲引擎設計,來發掘和打磨引擎內部功能,並學習各種程式設計架構,使可讀性增加便於維護。除此之外,組員們從中了解組件其中之原理並留下紀錄,供後人想了解遊戲引擎背後的原理參考。

\newpage